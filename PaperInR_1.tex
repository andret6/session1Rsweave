\documentclass[11pt]{article}

\title{Appendix Three: Adminstrative Data Technical Notes}
\author{
        Andrew Taylor\\
        Evans School of Public Policy and Governance\\
        University of Washington\\
        Seattle, WA 98115, \underline{United States}\\
        \texttt{andret6@uw.edu}
}
\date{\today}


\usepackage{Sweave}
\begin{document}
\Sconcordance{concordance:PaperInR_1.tex:PaperInR_1.Rnw:%
1 13 1 1 0 39 1}


\maketitle


\begin{abstract}
According to WikiPedia, the conventional definition of 'Data Science' is an "an interdisciplinary field of scientificis methods, processes, algorithms and systmes to extract knowledge or insights from data in various forms." In the popular world of living human beings, 'data science' is a buzz word that means anything from "please hire me I'm straight out of college" to "I have a PhD in compuer science AND statistics." This paper provides a overview of how some very basic data science skills were able to take what would have been prohibitivly unclean adminstrative court data and convert it into data sufficient for statistical analysis.

\end{abstract}

\section{Introduction}\label{intro}


\section{Explaining Labels}\label{outline}

Sections may use a label\footnote{In fact, you can have a label wherever you think a future reference to that content might be needed.}. This label is needed for referencing. For example the next section has label \emph{datas}, so you can reference it by writing: As we see in section \ref{datas}.

\section{Data analysis}\label{datas}

Here you can explain how to get the data.

\subsection{Exploration}\label{eda}

Here, I start exploring the data. 

\subsection{Modeling}\label{model}

Here I can propose a regression model.
\end{document}
